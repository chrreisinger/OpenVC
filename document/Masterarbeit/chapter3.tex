\chapter{Architektur}
{\em 
In diese Kapitel wird die Architektur erl�utert, dynamische sowie statische. Es wird dabei erkl�rt wie VHDL Source Code durch den Compiler verarbeitet wird.
}
\section{statische und dynamiche Architektur}
\section{Scanner/Parser mit Antlr}
\begin{quotation}
Eine Grammatik hei�t LL(1) (d.h. analysierbar von links nach rechts mit linkskanonischen Ableitungen und einem Vorgriffssymbol), wenn an jeder Stelle, an der man zwischen mehreren Alternativen w�hlen kann, gilt, dass die terminalen Anf�nge dieser Alternativen paarweise disjunkt sind. Mit anderen Worten: Der Parser muss jederzeit mit einem einzigen Vorgriffssymbol entscheiden k�nnen, welche von mehreren m�glichen Alternativen er w�hlen soll. \cite{moess}
\end{quotation}

Die im Standard \cite{ieee} beschriebene Syntax liegt bereits in Extended Backus-Naur Form (EBNF) \cite{wirth} vor, weist aber einige LL(1) Konflikte auf, die vorher behoben wurden.
In der Beschreibung von Coco/R [M�ss03] werden drei verschiedene M�glichkeiten erl�utert, wie diese Konflikte gel�st werden k�nnen. Mit Hilfe dieser drei Varianten ist es auch gelungen, die Grammatik in eine f�r Antlr valide LL(1) Form umzuwandeln. Die drei Ans�tze zur Behebung der Konflikte werden nachstehend erl�utert.

\subsection{Faktorisierung}
Bei der Faktorisierung werden gemeinsame Teile herausgezogen und an den Anfang der Produktion gestellt. Z.B. kann die folgende Produktion
\begin{center} A = a b c | a b d. \end{center} 
in die Produktion A'
\begin{center} A' = a b (c | d). \end{center} 
ohne Konflikte umgewandelt werden.
\subsection{Umwandlung in Iterationen}
Linksrekursion stellt in LL(k) Sprachen im Gegensatz zu LR basierten immer ein Problem dar. In der Produktion
\begin{center} A = A b | c. \end{center} 
starten beide Alternativen mit c. Durch eine Umwandlung der Rekursion in eine Iteration kann dieses Problem gel�st werden, z.B wird die Produktion A zu
\begin{center} A' = c \{b\}. \end{center} 
\subsection{Einsatz von \textit{Conflict Resolvers}}
Bei dem Einsatz von \textit{Conflict Resolvers} kann unterschieden werden, wie viele Tokens der Parser vorausschauen muss, um eine Entscheidung zu treffen.

\subsubsection{Konstante Anzahl von \textit{Lookahead Tokens}}
Diese F�lle k�nnten meistens auch durch den Einsatz von Faktorisierung gel�st werden, aber die Lesbarkeit der Grammatik w�rde darunter leiden. Durch die Ber�cksichtigung von mehreren Tokens in die Entscheidung verh�lt sich der Parser effektiv wie ein LL(k) basierter.

\subsubsection{Unbekannte Anzahl von \textit{Lookahead Tokens}}
Hier muss der Parser beliebig viele Tokens konsumieren und wird dabei in einen LL(*) basierten umgewandelt. [Par07]

\section{AST}
\section{Symboltabelle}
\section{Codeerzeugung}