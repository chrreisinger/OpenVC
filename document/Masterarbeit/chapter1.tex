\chapter{Aufgabenstellung und Motivation}
{\em 
Diese Kapitel gibt einen kurzen �berblick �ber die Aufgabenstellung und die Motivation der Masterarbeit.
Zus�tzlich wird wird eine �bersicht �ber die verschiedenen Themen die in den einzelnen Kaptitel beschrieben werden, gegeben.
}

Die in VHDL beschriebenen Schaltungen m�ssen vor der Logiksynthese am Computer simuliert werden, um sie auf m�gliche Fehler zu �berpr�fen. Die dazu ben�tigten Software-Werkzeuge sind bis auf eine Ausnahme nur von kommerziellen Anbietern erh�ltlich und f�r Studenten gar nicht oder nur f�r in der Gr��e beschr�nkte Schaltungen erh�ltlich.

\section{Aufbau}
Diese Masterabeit ist wie folgt aufgebaut:
\begin{itemize}
	\item In Kapitel 2 wird eine Einf�hrung in VHDL gegeben.
	\item Kapitel 3 er�rtert die Architektur des Compilers
  \item Kapitel 4 wird der gew�hlte Ansatz mit anderen Compilern verglichen
  \item Kapitel 5 fasst die Arbeit zusammen und gibt einen Ausblick auf weitere m�gliche Verbesserungen
\end{itemize}
