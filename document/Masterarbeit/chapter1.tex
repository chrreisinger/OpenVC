\chapter{Aufgabenstellung und Motivation}
{\em 
Diese Kapitel gibt einen kurzen �berblick �ber die Aufgabenstellung und die Motivation der Masterarbeit.
Zus�tzlich wird wird eine �bersicht �ber die verschiedenen Themen die in den einzelnen Kaptitel beschrieben werden, gegeben.
}

Die in VHDL beschriebenen Schaltungen m�ssen vor der Logiksynthese am Computer simuliert werden, um sie auf m�gliche Fehler zu �berpr�fen. Die dazu ben�tigten Software-Werkzeuge sind bis auf eine Ausnahme nur von kommerziellen Anbietern erh�ltlich und f�r Studenten gar nicht oder nur f�r in der Gr��e beschr�nkte Schaltungen erh�ltlich.

Diese Einschr�nkungen waren die Motivation dieser Arbeit und das Ziel dieser ist es, einen Compiler f�r einen ausreichend gro�en Teilbereich des VHDL 2002 Standards zu implementieren, um damit reale Chip-Designs erfolgreich simulieren zu k�nnen.

Folgende Probleme sind dabei zu l�sen:
\begin{itemize}
\item Es muss ein Parser erstellt werden und es soll darin durch semantische Aktionen ein Abstrakter Syntaxbaum erzeugt werden, der in weiteren Verarbeitungsschritten manipuliert werden kann.
\item Im n�chsten Schritt muss der erzeugte AST traversiert werden um die Symboltabelle zu verwalten damit die n�tigen Typ�berpr�fungen durchgef�hrt werden k�nnen. Es soll auch darauf geachtet werden, dass f�r den Benutzer des Compilers gute und hilfreiche Fehlermeldungen erzeugt werden.
\item Anschlie�end muss der erzeugte Zwischencode in JVM-Bytecode �bersetzt werden. Dabei ist zu beachten, dass die Semantik der Sprache nicht verloren geht und wie eventuelle Beschr�nkungen der JVM zu umgehen sind (z.B out Parameter).
\end{itemize}

\section{Aufbau}
Diese Masterabeit ist wie folgt aufgebaut:
\begin{itemize}
	\item In Kapitel 2 wird eine Einf�hrung in VHDL gegeben.
	\item Kapitel 3 er�rtert die Architektur des Compilers
  \item Kapitel 4 wird der gew�hlte Ansatz mit anderen Compilern verglichen
  \item Kapitel 5 fasst die Arbeit zusammen und gibt einen Ausblick auf weitere m�gliche Verbesserungen
\end{itemize}
