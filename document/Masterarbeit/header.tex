\documentclass[german,12pt,a4paper,oneside]{report}
\usepackage{hyperref} %sorgt f�r links im erzeugten pdf
\usepackage[english,ngerman]{babel}
\usepackage{listings} %wird f�r code listings verwendet
\usepackage{color}
\usepackage{pdfpages} %wird f�r includepdf verwendet
\usepackage{graphicx} %wird f�r includeimage verwendet
%\usepackage[xindy,toc]{glossaries}
\usepackage[T1]{fontenc} %damit sonderzeichen wie | richtig angezeigt werden
\usepackage[latin1]{inputenc} %damit Umlaute verwendet werden k�nnen
\usepackage{fancyhdr} %Packet f�r fany headers
\usepackage{todonotes} %Packet f�r TODOs
\usepackage{setspace} %f�r Zeilenabstand
\usepackage{sectsty} %damit der Abstand der �berschriften ver�ndert werden kann
\usepackage{datetime} %f�r \monthname{}
\usepackage{titletoc} % Inhaltsverzeichnis anpassen
\usepackage{mdwlist} %sorgt f�r normale Textabst�nde in der Itemize-Umgebung
%does not work \usepackage{underscore} %damit man underscrote ohne escape zeichen verwenden kann
 
\definecolor{dkgreen}{rgb}{0,0.6,0}
\definecolor{gray}{rgb}{0.5,0.5,0.5}
\definecolor{mauve}{rgb}{0.58,0,0.82}
 
\lstset{ %
  language=VHDL,                  % the language of the code
 % basicstyle=\footnotesize,       % the size of the fonts that are used for the code
  basicstyle=\small,              % print whole listing small
  numbers=left,                   % where to put the line-numbers
  numberstyle=\tiny\color{gray},  % the style that is used for the line-numbers
  stepnumber=1,                   % the step between two line-numbers. If it's 1, each line will be numbered
  numbersep=5pt,                  % how far the line-numbers are from the code
  backgroundcolor=\color{white},  % choose the background color. You must add \usepackage{color}
  showspaces=false,               % show spaces adding particular underscores
  showstringspaces=false,         % underline spaces within strings
  showtabs=false,                 % show tabs within strings adding particular underscores
  frame=single,                   % adds a frame around the code
  framexleftmargin=5mm,           % margin adujsted to that line-numer is inside of the frame
  rulecolor=\color{black},        % if not set, the frame-color may be changed on line-breaks within not-black text (e.g. commens (green here))
  tabsize=2,                      % sets default tabsize to 2 spaces
  captionpos=b,                   % sets the caption-position to bottom
  breaklines=true,                % sets automatic line breaking
  breakatwhitespace=false,        % sets if automatic breaks should only happen at whitespace
  title=\lstname,                 % show the filename of files included with \lstinputlisting;
                                  % also try caption instead of title
  keywordstyle=\color{blue},      % keyword style
  commentstyle=\color{dkgreen},   % comment style
  stringstyle=\color{mauve}%,      % string literal style
%  escapeinside={\%*}{*)},         % if you want to add a comment within your code
%  morekeywords={*,...}            % if you want to add more keywords to the set
}
% taken from: http://tex.stackexchange.com/questions/42030/using-listings-package-to-colorize-the-source-code-of-antlr-grammar-file
\lstdefinestyle{ANTLR}{
    %basicstyle=\small\ttfamily\color{magenta},%
    %breaklines=true,%                                      allow line breaks
    moredelim=[s][\color{green!50!black}\ttfamily]{'}{'},% single quotes in green
    moredelim=*[s][\color{black}\ttfamily]{options}{\}},%  options in black (until trailing })
    commentstyle={\color{gray}\itshape},%                  gray italics for comments
    morecomment=[l]{//},%                                  define // comment
    emph={%
        STRING%                                            literal strings listed here
        },emphstyle={\color{blue}\ttfamily},%              and formatted in blue
    alsoletter={:,|,;},%
    morekeywords={:,|,;},%                                 define the special characters
    %keywordstyle={\color{black}},%                         and format them in black
}

% �� Start: Angaben zur Formatierung von �berschriften �� %
%from http://tom.hirschvogel.org/blog/latex-4-abstaende-im-inhaltsverzeichnis-festlegen/
\titlecontents{chapter}[1.5em]{\addvspace{1pc}\bfseries}{\contentslabel{1.5em}}
 {\hspace*{-1.5em}}{\titlerule*[0.3pc]{.}\contentspage}
\titlecontents{section}[3.7em]{}{\contentslabel{2.2em}}{}
 {\titlerule*[0.3pc]{.}\contentspage}
\titlecontents{subsection}[6.7em]{}{\contentslabel{2.95em}}{}
 {\titlerule*[0.3pc]{.}\contentspage}
\titlecontents{subsubsection}[10.6em]{}{\contentslabel{3.8em}}{}
 {\titlerule*[0.3pc]{.}\contentspage}
\titlecontents{paragraph}[15.25em]{}{\contentslabel{4.6em}}{}
 {\titlerule*[0.3pc]{.}\contentspage} 

\setcounter{tocdepth}{4}
\setcounter{secnumdepth}{4}
 % �� Ende: Angaben zur Formatierung von �berschriften �� %

\setlength\parindent{0pt} %Schaltet das automatische Einr�cken f�r einen neuen Absatz aus
%\selectlanguage{\german} %deutsch als Sprache f�r �berschriften
%\pagestyle{headings} %Kapitel�berschrift und Seitenzahl in der Kopfzeile
%\pagestyle{fancy} %fancy header and footer

%�berschrift nach links schieben
\sectionfont{\hspace{-1 cm}}
\subsectionfont{\hspace{-1 cm}}
%\subsubsectionfont{\hspace{-1 cm}}

%plain wird neu definiert, damit es auch bei chapter funktionert, da diese default m��ig \pagestyle{plain} verwenden
\fancypagestyle{plain}{%
\fancyhead{}
\fancyfoot{}
\fancyhead[L]{\leftmark}
\fancyhead[R]{\thepage}
\setlength{\headheight}{15pt}
\renewcommand{\headrulewidth}{0.4pt} %obere Trennlinie
%\renewcommand{\footrulewidth}{0.4pt} %untere Trennlinie
}

%Legt die Art des Seitenformats f�r alle nachfolgenden Seiten fest. Die Voreinstellung plain steht f�r eine zentrierte Seitennummer am Seitenfu�. Durch empty erreicht man Seiten, die keinerlei Kopf oder Fu� enthalten. headings erzeugt eine Kopfzeile aus der Seitennummer und der �berschrift des laufenden Abschnitts. myheadings schlie�lich erlaubt es, den Seitenkopf selbst mit den markright und markboth-Befehlen zu definieren.
\pagestyle{plain}
%default format \renewcommand{\chaptermark}[1]{\markboth{\MakeUppercase{\chaptername\ \thechapter.\ #1}}{}}
\renewcommand{\chaptermark}[1]{\markboth{\textit{#1}}{}} %format vom text im header 
\newcommand{\myaddcontentsline}[3]{\begin{singlespace} \addcontentsline{#1}{#2}{\vspace{-2ex} #3} \end{singlespace}} %\addcontentsline{toc}{chapter}{Lebenslauf}

\renewcommand{\lstlistlistingname}{Quellcodeverzeichnis}
\renewcommand{\lstlistingname}{Quellcode}

%\section{Kapitel} 1
%\subsection{Unterkapitel} 1.1
%\subsubsection{Unterunterkapitel} 1.1.1
%\paragraph{Abschnitt} keine Numerierung kein Inhaltsverzeichniseintrag
%\subparagraph{Unterabschnitt} keine Numerierung kein Inhaltsverzeichniseintrag
%Anhang
%\begin{appendix}
%\section{Beispiel}
%\end{appendix}
%\footnote{1906 -- 1975}
%\todo{Add details.}
%Listings examples
%\lstinline[<Optionen>]{hier kommt der Quellcode...}
%\lstset{language=<Sprache>}
%\begin{lstlisting}[<Optionen>]
%hier kommt dann der Quellcode...
%\end{lstlisting}