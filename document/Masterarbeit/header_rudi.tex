
%\documentclass[11pt,a4paper,austrian,titlepage,
%chapterprefix,headsepline,parskip,pdftex,
%,pointlessnumbers,bibtotoc]{scrreprt}

%mru: removed 'austrian' option
\documentclass[12pt, a4paper, titlepage, headsepline, pdftex, parskip , bibtotoc, %pointlessnumbers%
]{scrreprt}

\usepackage{varioref}
\usepackage{zref-abspage}
\usepackage{datetime}

\makeatletter
\newcommand*{\HERE}{%
\zref@label{HERE}%
}

\makeatletter
\AtBeginDocument{%
\hypersetup{%
pdfstartpage=\zref@extractdefault{HERE}{abspage}{1 }%
}%
}
\makeatother
% \usepackage{hyperref}




%%% Absätze bei tieferen Ebenen einschalten
\makeatletter %% Sonderbedeutung von @ aufheben
\renewcommand{\paragraph}{\@startsection
   {paragraph} % name
   {4} % ebene
   {0mm} % einzug
   {-\baselineskip} % vorabstand
   {0.1\baselineskip} % nachabstand
   {\normalfont\normalsize\bfseries}} % stil
\makeatother %% Sonderbedeutung von @ wieder

\makeatletter %% Sonderbedeutung von @ aufheben
\renewcommand{\subparagraph}{\@startsection
   {subparagraph} % name
   {5} % ebene
   {0mm} % einzug
   {-\baselineskip} % vorabstand
   {0.1\baselineskip} % nachabstand
   {\normalfont\normalsize\bfseries}} % stil
\makeatother %% Sonderbedeutung von @ wieder

\usepackage{setspace}
\onehalfspacing

%does not work \usepackage{underscore} %damit man underscrote ohne escape zeichen verwenden kann
\usepackage{mdwlist} %sorgt für normale Textabstände in der Itemize-Umgebung
\usepackage[pdftex]{graphicx}

% for colours
\usepackage[pdftex]{color}

\usepackage[colorlinks=true,
    linkcolor=black,
    citecolor=black,
    urlcolor=black,
    breaklinks=true,
    bookmarksnumbered=true,
    hypertexnames=false,
    pdfpagemode=UseOutlines,
    pdfview=FitH,
    plainpages=false,
    pdfpagelabels,
    bookmarks=true,
    linktocpage=true]{hyperref}

\hypersetup{pdfauthor={Rudolf Mühlbauer},
    pdftitle={Design und Implementierung des OpenVC VHDL Compilers},
    pdfsubject={Mastersthesis},
    pdfkeywords={},
    pdfcreator={pdfLaTeX with hyperref (\today})}

%%% for Source-Code
\usepackage{listings}
\renewcommand{\lstlistlistingname}{Quellcodeverzeichnis}
\renewcommand{\lstlistingname}{Quellcode}

\definecolor{dkgreen}{rgb}{0,0.6,0}
\definecolor{gray}{rgb}{0.5,0.5,0.5}
\definecolor{mauve}{rgb}{0.58,0,0.82}
\lstset{ %
  language=VHDL,                  % the language of the code
  %basicstyle=\footnotesize,      % the size of the fonts that are used for the code
  basicstyle=\small,              % print whole listing small
  numbers=left,                   % where to put the line-numbers
  numberstyle=\tiny\color{gray},  % the style that is used for the line-numbers
  stepnumber=1,                   % the step between two line-numbers. If it's 1, each line will be numbered
  numbersep=5pt,                  % how far the line-numbers are from the code
  backgroundcolor=\color{white},  % choose the background color. You must add \usepackage{color}
  showspaces=false,               % show spaces adding particular underscores
  showstringspaces=false,         % underline spaces within strings
  showtabs=false,                 % show tabs within strings adding particular underscores
  frame=single,                   % adds a frame around the code
  framexleftmargin=5mm,           % margin adujsted to that line-numer is inside of the frame
  rulecolor=\color{black},        % if not set, the frame-color may be changed on line-breaks within not-black text (e.g. commens (green here))
  tabsize=2,                      % sets default tabsize to 2 spaces
  captionpos=b,                   % sets the caption-position to bottom
  breaklines=true,                % sets automatic line breaking
  breakatwhitespace=false,        % sets if automatic breaks should only happen at whitespace
  title=\lstname,                 % show the filename of files included with \lstinputlisting;
                                  % also try caption instead of title
  keywordstyle=\color{blue},      % keyword style
  commentstyle=\color{dkgreen},   % comment style
  stringstyle=\color{mauve}%,      % string literal style
%  escapeinside={\%*}{*)},         % if you want to add a comment within your code
%  morekeywords={*,...}            % if you want to add more keywords to the set
}
% taken from: http://tex.stackexchange.com/questions/42030/using-listings-package-to-colorize-the-source-code-of-antlr-grammar-file
\lstdefinestyle{ANTLR}{
    %basicstyle=\small\ttfamily\color{magenta},%
    %breaklines=true,%                                      allow line breaks
    moredelim=[s][\color{green!50!black}\ttfamily]{'}{'},% single quotes in green
    moredelim=*[s][\color{black}\ttfamily]{options}{\}},%  options in black (until trailing })
    commentstyle={\color{gray}\itshape},%                  gray italics for comments
    morecomment=[l]{//},%                                  define // comment
    emph={%
        STRING%                                            literal strings listed here
        },emphstyle={\color{blue}\ttfamily},%              and formatted in blue
    alsoletter={:,|,;},%
    morekeywords={:,|,;},%                                 define the special characters
    %keywordstyle={\color{black}},%                         and format them in black
}

% To-Do Command
\newcommand{\todo}[1]{\textcolor{red}{\textbf{ToDo:} #1}}

% ^th
\newcommand{\thup}[0]{$^{\text{th}}$\;}

% tabref and figref
\newcommand{\tabref}[1]{Table~\ref{#1}}
\newcommand{\figref}[1]{Figure~\ref{#1}}

% Internal-Link-Command
\newcommand{\internerLink}[1]{\hyperref[#1]
{See \ref*{#1}~\nameref{#1} auf S.~\pageref{#1}}}

% Internal-Link-Command 2
\newcommand{\ffinternerLink}[1]{\hyperref[#1]
{See p.~\pageref{#1}ff}}

% Internal-Link-Command x
\newcommand{\xinternerLink}[1]{\hyperref[#1]
{\ref*{#1}~\nameref{#1} auf S.~\pageref{#1}}}


%%% Continous Footnote
\newcounter{cfootnotecounter}
\newcommand{\cfootnote}[1]{\stepcounter{cfootnotecounter}
\footnote[\value{cfootnotecounter}]{#1}}

%%% Bild Befehle
\newcommand{\bild}[3]{\begin{figure} \begin{center}
\includegraphics{#1}
\caption{#2} \label{#3} \end{center} \end{figure}}

\newcommand{\bildH}[3]{\begin{figure}[h] \begin{center}
\includegraphics{#1}
\caption{#2} \label{#3} \end{center} \end{figure}}

\newcommand{\bildHS}[4]{\begin{figure}[ht] \begin{center}
\includegraphics[scale=#4]{#1}
\caption{#2} \label{#3} \end{center} \end{figure}}

\newcommand{\bildHTB}[4]{\begin{figure}[htb] \begin{center}
\includegraphics[scale=#4]{#1}
\caption{#2} \label{#3} \end{center} \end{figure}}

\newcommand{\bildTabelle}[3]{\begin{table}[htb] \begin{center}
\includegraphics{#1}
\caption{#2} \label{#3} \end{center} \end{table}}

\newcommand{\bildE}[5]{\begin{figure}[hb] \begin{center}
\includegraphics[height=#2, angle=#3]{#1}
\caption{#4} \label{#5} \end{center} \end{figure}}

\newcommand{\tabelle}[3]{\begin{table}[htb] \begin{center}
\input{#1}
\caption{#2} \label{#3} \end{center} \end{table}}

\flushbottom

% Change Page Settings
\setlength{\hoffset}{0mm} \setlength{\voffset}{0mm}
\setlength{\evensidemargin}{14.6mm}
\setlength{\oddsidemargin}{14.6mm} \setlength{\topmargin}{-20mm}
\setlength{\headheight}{15mm} \setlength{\headsep}{9mm}
\setlength{\textheight}{242mm} \setlength{\textwidth}{145mm}
\setlength{\footskip}{10mm}

%%% Obsolete due to parskip
%\setlength{\parskip}{3ex plus0.5ex minus0.5ex}
%\setlength{\parindent}{0mm}

%%% Margins of Float Commands
\setlength{\textfloatsep}{25pt plus5pt minus5pt}
\setlength{\intextsep}{25pt plus5pt minus5pt}

%%% Chapter Numbers at the side-margin
\renewcommand*{\othersectionlevelsformat}[1]{%
\llap{\csname the#1\endcsname\autodot\enskip}}

%%% In Header: Chapter title without Numbering
\renewcommand*{\chaptermarkformat}{}

%%% Format of chapter
\setkomafont{chapter}{\Huge}
\renewcommand*{\chapterformat}{\LARGE{\chapappifchapterprefix{\ }\thechapter\autodot\enskip}}

%%% Header
\usepackage[automark]{scrpage2}

\clearscrheadings \clearscrplain \clearscrheadfoot
\pagestyle{scrheadings}
\ohead{\pagemark}
\ihead{\headmark}
\cfoot{}

%%% Format of chapter header-pages
\renewcommand*{\chapterpagestyle}{scrheadings}

%% Structure of TOC
\setcounter{tocdepth}{\subsubsectionlevel}
\setcounter{secnumdepth}{\paragraphlevel}

%%% Array for tables
\usepackage{array}

%%% Fonts
\addtokomafont{chapter}{\sffamily}
\addtokomafont{sectioning}{\rmfamily}

% Language
%\usepackage[german,ngerman]{babel}
\usepackage[english,ngerman]{babel}
% Umlauten
\usepackage[latin1]{inputenc}
%\usepackage[utf8]{inputenc}
% T1 Fonts
\usepackage[T1]{fontenc}
\usepackage{ae}

% URLs
%\usepackage{url}
\usepackage{hyperref}

%%% "Schusterjungen" & "Hurenkinder"
\clubpenalty = 10000
\widowpenalty = 10000 \displaywidowpenalty = 10000


%%% PDF-page includes
\usepackage{pdfpages}

%%% Math
\usepackage{amsmath}
\usepackage{amssymb}

%%% Music
%\usepackage{musixtex}

%%% Sub-Figures
\usepackage{subfigure}
%\usepackage{subfig} % newer package

%%% Connected rows and cols
\usepackage{multirow}

%%% rotated text
\usepackage{rotating}






\graphicspath{{../images/}}
\DeclareMathOperator{\lum}{lum}
\def\CPP{{C\kern-.05em\raise.23ex\hbox{+\kern-.05em+}}}


%%% enable [H] option for floats
\usepackage{float}
\restylefloat{figure}
\restylefloat{table}

%
%
% LaTeX-Bilder einbinden
%
% \dalatexpic{bild.tex}{LABEL}{Eintrag Inhaltsverzeichnis}{Bildunterschrift}
%
\newcommand{\dalatexpic}[4]
{
  \begin{figure*}[!ht]
    \begin{center}
      \input{#1}
    \end{center}
    \ifthenelse{\equal{#3}{}}
               {\caption{{\label{#2}\textit{#3}}}}
               {\caption[#3]{{\label{#2}\textit{#4}}}}
  \end{figure*}
}
%
% EPS/PDF-Bilder einbinden
%
% \dapdfpic[Optionen]{bild.eps|pdf}{LABEL}{Eintrag Inhaltsverzeichnis}{Bildunterschrift}
%

\newcommand{\dapic}[5][]
{
  \begin{figure*}[!ht]
    \begin{center}
      \ifthenelse{\equal{#1}{}}
                 {\includegraphics{#2}}
                 {\includegraphics[#1]{#2}}
    \end{center}
    \ifthenelse{\equal{#4}{}}
               {\caption{{\label{#3}\textit{#5}}}}
               {\caption[#4]{{\label{#3}\textit{#5}}}}
  \end{figure*}
}


\newcommand{\rarr}{\ensuremath{\rightarrow}}


\usepackage{siunitx}
