\chapter{VHDL}
{\em 
In diese Kapitel wird ein �berblick �ber die Sprache VHDL geben. Zuerst wird die Geschichte und Standards erl�utert, anschlie�end die Sprache selbst.
}
\section{Geschichte}
VHDL (Very High Speed Integrated Circuit Hardware Description Language oder auch VHSIC Hardware Description Language) ist eine vom IEEE standardisierte Sprache zur Beschreibung digitaler Schaltungen, die von Hardware Designern f�r aktuelle Designs verwendet wird.

VDHL ist neben Verilog eine der weltweit am meisten genutzten Hardwarebeschreibungssprachen und hat sich in Europa zum "Quasi-Standard" entwickelt. Die erste Spezifikation der Sprache wurde in den fr�hen 1980er Jahren entwickelt und ist das Ergebnis von Normierungsbestrebungen eines Komitees, in dem die meisten gr��eren CAD-Anbieter und CAD-Nutzer, aber auch Vereinigungen wie die IEEE, vertreten waren. Der gr��te nordamerikanische Anwender, das US-Verteidigungsministerium, hat VHDL zum Durchbruch verholfen, indem es die Einhaltung der Syntax von VHDL als notwendige Voraussetzung f�r die Erteilung von Auftr�gen gemacht hat. \cite{ash}

VHDL was originally developed at the behest of the U.S Department of Defense in order to document the behavior of the ASICs that supplier companies were including in equipment. That is to say, VHDL was developed as an alternative to huge, complex manuals which were subject to implementation-specific details.

The idea of being able to simulate this documentation was so obviously attractive that logic simulators were developed that could read the VHDL files. The next step was the development of logic synthesis tools that read the VHDL, and output a definition of the physical implementation of the circuit.

Due to the Department of Defense requiring as much of the syntax as possible to be based on Ada, in order to avoid re-inventing concepts that had already been thoroughly tested in the development of Ada,[citation needed] VHDL borrows heavily from the Ada programming language in both concepts and syntax.

\subsection{Versionen und Standards}
Die Initiale Version von VHDL wurde unter dem IEEE Standard 1076-1987 ver�ffentlicht, und enthielt verschiedene Datentypen, numerische (integer and real), logische (bit and boolean), character und time, und arrays von bit und character (bit\_vector string)

Danach wurde erkannt, dass sich mit den vorhanden Datentypen "multi-valued logic" wo ein Signal mehrere Zust�nde (undefiniert, schwach, stark) nicht modellieren lassen kann. Dieses Problem wurde mit dem IEEE 1164 Standard gel�st, der die zus�tzlichen 9-wertige logische Datentypen scalar std\_ulogic und std\_ulogic\_vector definiert.

Im Jahr 1993 wurde eine Aktualisierung des Standards ver�ffentlicht, der die Syntax konsistenter machte, den Zeichensatz auf ISO-8859-1 erweiterte und den XNOR Operator einf�hrte.

Kleiner �nderungen des Standards im Jahr 2002 und 2002 f�gten VHDL unter anderem die schon den in Ada bekannten protected types hinzu, mit dem sich wechselseite Ausschl�sse definieren lassen k�nnen.

Neben dem IEEE Standard 1076, der die eigentliche Sprache definiert, gibt es zus�tzlich noch verwandte die Sprache erweitern oder Funktionalit�t von verschiedenen Libraries beschreibt. Der Standard 1076.2 f�gt mehr added better handling of real and complex data types, 1076.3 f�hrt signed und unsigned Datentypen ein, mit denen arithmetische Operationen auf arrays ausgef�hrt werden k�nnen. IEEE Standard 1076.1 (besser bekannt als VHDL-AMS) beschreibt Erweiterungen f�r gemischte analog und digitale Schaltungen.

Im Juni 2006, wurde durch eine technisches Komittee ein Enwurf f�r VHDL-2006 vorgelegt. Dieser Entwurf war vollkommen abw�rtskompatibel mit �lteren Versionen, f�gte aber viele Erweiterungen hinzu die das Schreiben von Code erleichterten. Die wichtigste �nderung war, dass die verwandten Standards 1164, 1076.2, 1076.3 in den 1076 Standard hinzugef�gt wurden. Zus�tzlich wurden neue Operatoren , flexiblere Syntax f�r case und generate Statements und es wurde ein Interface zu C/C++ hinzugef�gt.

Im Jahr 2008 wurde eine neuere Version dieses Entwurf, nachdem die bisher darin entdecken Probleme gel�st wurden, ver�ffentlicht. Anschlie�en wurde die letzte Version als IEEE 1076-2008 Standard im Januar 2009 ver�ffentlicht.
\section{Sprachumfang}
\lstset{caption={Beispiel f�r ein AND Gate},label=ANDGATE}
\lstinputlisting{src/ANDGATE.vhd}