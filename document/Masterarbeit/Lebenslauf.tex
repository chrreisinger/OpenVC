%% start of file `template.tex'.
%% Copyright 2006-2012 Xavier Danaux (xdanaux@gmail.com).
%
% This work may be distributed and/or modified under the
% conditions of the LaTeX Project Public License version 1.3c,
% available at http://www.latex-project.org/lppl/.


\documentclass[11pt,a4paper,sans]{moderncv}   % possible options include font size ('10pt', '11pt' and '12pt'), paper size ('a4paper', 'letterpaper', 'a5paper', 'legalpaper', 'executivepaper' and 'landscape') and font family ('sans' and 'roman')

% moderncv themes
\moderncvstyle{casual}                        % style options are 'casual' (default), 'classic', 'oldstyle' and 'banking'
\moderncvcolor{blue}                          % color options 'blue' (default), 'orange', 'green', 'red', 'purple', 'grey' and 'black'
%\renewcommand{\familydefault}{\sfdefault}    % to set the default font; use '\sfdefault' for the default sans serif font, '\rmdefault' for the default roman one, or any tex font name
%\nopagenumbers{}                             % uncomment to suppress automatic page numbering for CVs longer than one page

% character encoding
%\usepackage[utf8]{inputenc}                  % if you are not using xelatex ou lualatex, replace by the encoding you are using

% adjust the page margins
\usepackage[scale=0.75]{geometry}
\setlength{\hintscolumnwidth}{3.5cm}           % if you want to change the width of the column with the dates
%\setlength{\maketitlenamewidth}{10cm}        % for the 'classic' style, if you want to force the width allocated to your name and avoid line breaks. be careful though, the length is normally calculated to avoid any overlap with your personal info; use this at your own typographical risks...

% personal data
\firstname{Christian}
\familyname{Reisinger}
\title{Bakk.techn.}               % optional, remove the line if not wanted
\address{Z�lowstra{\ss}e 5/1}{4040 Linz}    % optional, remove the line if not wanted
\mobile{0664/1717647}                     % optional, remove the line if not wanted
\email{chr\_reisinger@yahoo.de}                          % optional, remove the line if not wanted
%\extrainfo{additional information}            % optional, remove the line if not wanted
%\photo[64pt][0.4pt]{picture}                  % '64pt' is the height the picture must be resized to, 0.4pt is the thickness of the frame around it (put it to 0pt for no frame) and 'picture' is the name of the picture file; optional, remove the line if not wanted
%\quote{Some quote (optional)}                 % optional, remove the line if not wanted

%----------------------------------------------------------------------------------
%            content
%----------------------------------------------------------------------------------
\begin{document}
%-----       resume       ---------------------------------------------------------
\makecvtitle

\section{Ausbildung}
%\cventry{year--year}{Degree}{Institution}{City}{\textit{Grade}}{Description}  % arguments 3 to 6 can be left empty

%\cventry{2011--2012}{}{Johannes Kepler Universit�t}{Linz}{}{Masterstudium}
%\cventry{2006--2011}{}{Johannes Kepler Universit�t}{Linz}{}{Bachelorstudium}
%\cventry{2000--2005}{}{HTBLA Leonding}{}{}{Zweig EDV und Organistaion Juni 2005: \newline{}Abschluss mit gutem Erfolg
%Project Award 2005 der HTBLA Leonding und Perg, Kategorie Diplomarbeit}
%\cventry{2005--2006}{}{Pr�senzdienst}{}{}{}

\cvitem{2011--2012}{Masterstudium an der Johannes Kepler Universit�t, Linz}
\cvitem{2006--2011}{Bachelorstudium an der Johannes Kepler Universit�t, Linz}
\cvitem{2005--2006}{Pr�senzdienst}
\cvitem{2000--2005}{HTBLA Leonding, Zweig EDV und Organistaion}

\section{Masterarbeit}
\cvitem{Titel}{\emph{Design und Implementierung eines VHDL Compilers}}
\cvitem{Betreuer}{Prof. Dr. Dr. h.c. Hanspeter M�ssenb�ck}
%\cvitem{Beschreibung}{Short thesis abstract}

\section{Arbeitserfahrung}
\subsection{Praktika}
\cvitem{Juli/Aug. 2004}{Siemens Austria, Wien}
\cvitem{Juli/Aug. 2005}{MIC Customs Solutions, Linz}
\cvitem{Juli - Sep. 2006}{Industrie Informatik, Linz}
\cvitem{Juli/Aug. 2007}{MIC Customs Solutions, Linz}
\subsection{Festanstellungen}
\cvitem{2011 - }{DMCE, Linz}

\section{Fremdsprachen}
\cvitem{Englisch}{flie{\ss}end, Wort und Schrift}{}
%\cvitemwithcomment{Language 2}{Skill level}{Comment}

\section{Computer Kenntnisse}
\cvitem{Programmiersprachen}{		C/C++, Java, C\#, Scala, VHDL, Python}
\cvitem{Tools}{Eclipse, Lint, Klocwork, Modelsim}
%\cvdoubleitem{category 2}{XXX, YYY, ZZZ}{category 5}{XXX, YYY, ZZZ}

\clearpage
\end{document}